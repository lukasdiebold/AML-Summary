\section{Regression}
\subsection{Act 1: High-dimensional regression is unstable}

we assume X and y are distributed according to a distribution $p_*$ (i.e. $X, y \sim p_*$) and 
$$
y = f_*(x) + \varepsilon \quad \text{with } \varepsilon \sim \mathcal N(0, \sigma)
$$
Then our task is to estimate $f_*$ from $D = \{x_i, y_i\} \sim p_*$. The problem in this form is not tractable, this is why we restrict the choice of functions. First we limit ourselves to linear functions, that is functions of the form 
$$f_*(x) = \beta^\top x$$
This problem we then solve using Maximum Likelihood Estimation (MLE), that is
$$
\begin{aligned}
\widehat{\beta} &=\arg \max _{\beta \in \mathbb{R}^R} p(D \mid \beta)\\
&= ... \\
&= \arg \min \text{MSE}(D, \beta) \\
&= \frac{1}{n} \sum_{i \leq n} (y_i - \beta ^\top x_i)^2\\
&= (X^\top X)^{-1} X^\top y \\
&= ...\\
&= X^\top (X X ^\top )^{-1}y
\end{aligned}
$$
which is known as the ordinary least squares estimator, where 
$$
X=\left[\begin{array}{c}
-x_1- \\
\vdots \\
-x_n-
\end{array}\right] \quad y=\left[\begin{array}{c}
y_1 \\
\vdots \\
y_n
\end{array}\right]
$$
This estimator has some interesting properties. It is unbiased and, by the Gauss-Markov Theorem, it is the best linear unbiased estimator (BLUE), i.e. it attains the smallest variance among all linear unbiased estimators.
Thus, from the formula
$$
\text{error} = \text{bias}^2 + \text{variance} + \text{noise}
$$
we find that this estimator is the one with the smallest error of all the unbiased estimators. Then why does no-one use this estimator? If we introduce a bit of bias, we can significantly reduce the variance. So what is $\operatorname{Var}(\hat b)$? For this we use the SVD of $X = UDV^\top$; if we plug this into the formula from above we get
$$\hat b = VD^{-1}U^\top y$$
so, since $y$ has a gaussian distribution and we just multiply it with a matrix ($VD^{-1}U^\top$), $\hat b$ also has a gaussian distribution. Then
$$\operatorname{Var}(\hat{\beta})=\ldots=\sigma^2 V D^{-2} V^{\top} =\sigma^2 \sum_{i \leq n} \frac{1}{D_{ii}^2} V_i V_i^{\top} $$
What are the implications of this? In high-dimensional data many features are correlated, so $X$ is close to low rank and several singular values $D_{ii}$ are tiny. The $1/D_{ii}^2$ factors then explode, making the variance (and with it prediction instability) very large, even though the estimator remains unbiased. 

\subsection{Act 2: Stability via Regularization}
The typical process of \textbf{Bayesian inference} goes through the follwing stages:
\begin{enumerate}
    \item Prior $\beta \sim \mathcal N(0, \tau^2 I)$
    \item Observe likelihood $p(D | \beta)$
    \item Posterior \textit{(adjustment of the likelihood)} $p(\beta | D) \propto p(\beta) p(D | \beta) \propto \exp \left(-\frac{1}{2 \sigma^2} \text{MSE}(D, \beta)-\frac{1}{2 \tau^2}\|\beta\|^2\right)$
\end{enumerate}
Observe that the last term is equivalent to the loss induced by \textbf{ridge regression}. With a Laplace prior (heavy tails) on $\beta$ we obtain \textbf{lasso} instead. The prior variance $\tau^2$ controls the amount of shrinkage and therefore the bias-variance trade-off. If we do the math we get
$$\hat{\beta}_{\text {MAP }}=\left(x^{\top} x+\frac{\sigma^2}{\tau^2} I\right)^{-1} x^{\top} y$$ and if we do again an SVD to calculate the variance we get
$$\text{Var}(\hat \beta_{\text {MAP }}) = \sigma^2 \sum_{i\leq n} \frac{D_{ii}^2}{\left(D_i^2+\frac{\sigma_i^2}{\tau^2}\right)} V_i V_i^{\top}$$
Compared to OLS, the shrinkage factor $\frac{D_{ii}^2}{D_i^2+\sigma_i^2/\tau^2}$ down-weights directions with small singular values, reducing variance at the cost of introducing bias. 

\subsection{Act 3: Polynomial regression via kernels}
Now we change our assumption for $f_*(x)$ so that it is a polynomial function on $x$. We have
$$f_*(x) = \varphi(x)^\top \beta _*$$
where $\beta_* \in \mathbb{R}^{\infty}$ and 
$$
\varphi(X)=K_x\left(\frac{x_1^{\alpha_1} ... x_d^{\alpha_d}}{\sqrt{\alpha_{1}!\ldots \alpha_{d}!}}\right)_{\alpha \in \mathbb{N}^d}
$$
Then for $x, x^{\prime} \in \mathbb{R}^a$
$$
\begin{aligned}
\varphi(x)^{\top} \varphi\left(x^{\prime}\right) & =K_{RBF}\left(x, x^{\prime}\right) \\
& =\exp \left(-\frac{1}{2} ||x-x^{\prime} ||^2\right)
\end{aligned}$$
and 
$$
\begin{aligned}
\hat{\beta} & =\arg \min _{\beta \in \mathbb{R}^{\infty}} \frac{1}{n} \sum_{i \leq n}\left(y_i-\varphi\left(x_i\right)^{\top} \beta\right)^2 \\
& =\Phi^{\top}\left(\Phi \Phi^{\top}\right)^{-1} y
\end{aligned}
$$
where 
$$
\Phi=\left[\begin{array}{l}
\varphi(x)^{\top}  \\
\varphi\left(x_2\right)^{\top}\\
\varphi\left(x_n\right)^{\top}
\end{array}\right] \in \mathbb R^{n \times \infty}
$$
This doesn't look good. But let $x_*$ be a test point. Then
$$
\begin{aligned}
\hat y_* & =\varphi\left(x_*\right)^{\top} \hat{\beta} \\
& =\varphi\left(x_*\right)^{\top} \Phi^{\top}\left(\Phi \Phi^{\top}\right)^{-1} y\\
&= k(x_*)^{\top} K^{-1} y
\end{aligned}
$$
where $k(x_*) = \left(\varphi\left(x_*\right)^{\top} \varphi\left(x_i\right)\right)_{1 \leq i \leq n}$ and $K_{ij}=\left(\varphi(x_i)^{\top} \varphi(x_j)\right)$. This is the kernel trick: we only need inner products via $K$, not the infinite-dimensional $\varphi(\cdot)$.

The problem is that the inversion of the matrix is $O(n^3)$, which becomes costly for large datasets even though we avoided the infinite feature map explicitly.

\subsection{Act 4: Neural Networks}
We assume $f_*$ has only a single, very wide hidden layer.
$$f_*(X)=\frac{1}{\sqrt{m}} \sum_{i \leq m} \alpha_i \phi\left(\omega_i^{\top} X\right)$$
where $\phi$ is a nonlinear activation function (e.g. ReLU, tanh), and the network has $m$ hidden units. The parameters are $\theta=\left\{\alpha_i, w_i\right\}_{i \leq m}$, i.e. both the output weights $\alpha_i$ and the input weights $w_i$ are learned. We initialize with
$$
\theta_0 \sim \mathcal N\left(0, w^2\right)
$$
and we update our parameters using gradient descent. 
$$
\theta_{t+1} \leftarrow \theta_t-\eta \nabla_\theta \text { MSE}(D, \theta_t)
$$
The gradient can be written in matrix form as
$$\nabla_\theta \operatorname{MSE}\left(D, \theta_t\right)=\widetilde{\Phi}_t^{\top}\left(f_t-y\right)$$
where 
$$
\tilde{\Phi}_t=\left(-\nabla_\theta f\left(x_i ; \theta_t\right)^\top-\right)_{i \leq n} \in \mathbb{R}^{n \times |\theta|} \quad \text{ and } \quad f_t=\left(f(x_i ; \theta_t)\right)_{i \leq n} \in \mathbb{R}^n
$$
Here $\tilde{\Phi}_t$ is the feature matrix whose $i$-th row is the gradient of the network output with respect to all parameters, evaluated at data point $x_i$ and current parameters $\theta_t$.

In the \textit{lazy training regime} (small learning rate, wide network), the parameters stay close to initialization, so we can linearize the network via a first-order Taylor expansion around $\theta_0$:
$$f_t \approx f_0+\tilde \Phi_0\left(\theta_t-\theta_0\right)$$
Assuming the feature matrix $\tilde{\Phi}_t$ remains approximately constant at $\tilde{\Phi}_0$ (which holds when $m \to \infty$), gradient flow yields
$$
\theta_t-\theta_0=\tilde{\Phi}_0^{\top}\left(\tilde{\Phi}_0 \tilde{\Phi}_0^{\top}\right)^{-1}\left(f_t-f_0\right)
$$
This says the parameter change lies in the span of the gradients and is chosen to optimally fit the training residuals.

Now let $x_*$ be a test point. Plugging the linearization into the prediction yields
$$
f_t\left(x_*\right) \approx f_0\left(x_*\right)+\nabla_\theta f\left(x_*, \theta_0\right)^{\top} \tilde{\Phi}_0^{\top}\left(\tilde{\Phi}_0 \tilde{\Phi}_0^{\top}\right)^{-1}\left(f_t-f_0\right)
$$
Define the \textit{neural tangent kernel (NTK)} $K$ with entries
$$
K_{ij} = \nabla_\theta f(x_i, \theta_0)^{\top} \nabla_\theta f(x_j, \theta_0) = \left[\tilde{\Phi}_0 \tilde{\Phi}_0^{\top}\right]_{ij}
$$
and similarly $k(x_*) = \left(\nabla_\theta f(x_*, \theta_0)^{\top} \nabla_\theta f(x_i, \theta_0)\right)_{i \leq n}$. 

In the infinite-width limit ($m \to \infty$), the random initialization ensures $f_0(x) \to 0$ for all $x$ (the outputs average out), and after infinite training time ($t \to \infty$), gradient descent drives the training residual to zero so $f_t \to y$. The prediction becomes
$$
f_\infty\left(x_*\right) = k\left(x_*\right)^{\top} K^{-1} y
$$
This is exactly the result we obtained in Act 3 for kernel regression. 

\textbf{Conclusion:} Gradient descent on a very wide neural network operates in a \textit{kernel regime}, where training is equivalent to kernel ridge regression with the neural tangent kernel. The NTK is determined by the architecture and activation function, but the solution has the same closed-form structure $k(x_*)^{\top} K^{-1} y$ as any other kernel method. In practice, finite-width networks can escape this regime and learn richer, feature-learning representations—this lazy limit is a useful theoretical baseline.